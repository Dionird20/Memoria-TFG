%%%%%%%%%%%%%%%%%%%%%%%%%%%%%%%%%%%%%%%%%%%
% Página de resumen del proyecto en inglés%
%%%%%%%%%%%%%%%%%%%%%%%%%%%%%%%%%%%%%%%%%%%
% !TEX root = A0.MiTFG.tex

\pagestyle{fancy}
\addstarredchapter{Abstract}

\begin{center}
	\scshape
	E.T.S. de Ingeniería de Telecomunicación, Universidad de Málaga
\end{center}

\bigskip

\begin{center}
	\Large \scshape
	\textbf{\tfgtitlenameENG}
\end{center}

\bigskip \bigskip \bigskip

\begin{minipage}{\textwidth}

\textbf{Author:} \tfgauthorname

\medskip

\textbf{Supervisor:} \tfgtutorname

\medskip

\textbf{Co-supervisor:} José Manuel Cano García

\medskip

\textbf{Department:} Departamento Tecnología Electrónica

\medskip

\textbf{Degree:} Grado en Ingeniería de sistemas electrónicos

\medskip

\textbf{Keywords:} Cardiac event recorder, electrocardiography, ECG, Real-time QRS detection, Qt, Tiva C Series (TM4C123GXL).

\bigskip \bigskip


\end{minipage}

\begin{center}
	\textbf{Abstract}
\end{center}

Monitoring anomalies in the electrocardiography (ECG) recording of patients with cardiac pathologies in real time, is the basis of the operation of cardiac event monitors. Therefore, they must be designed both to minimize their consumption and cost, as well as to present high reliability in the real-time detection of cardiac abnormalities.

Given that, normally, the previous step for the classification of an ECG segment as abnormal, is the detection of the heartbeats contained in said segment. In this Degree Final Project a test environment has been developed to perform efficacy and reliability of heartbeat detection algorithms in real time, creating a functional prototype following the guidelines of the UNE-EN 60601-2-47 standard for testing portable heart monitors.

The validation of the system has been done using a basic beat detection algorithm, also implemented in this DFP, and the MIT-BIH ECG database.

\blankpage
