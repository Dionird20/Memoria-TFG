%%%%%%%%%%%%%%%%%%%%%%%%%%%%%%%%%%%%%%%%%%%
% Página de resumen del proyecto en inglés%
%%%%%%%%%%%%%%%%%%%%%%%%%%%%%%%%%%%%%%%%%%%
% !TEX root = A0.MiTFG.tex

\pagestyle{fancy}
\addstarredchapter{Abstract}

\begin{center}
	\scshape
	E.T.S. de Ingeniería de Telecomunicación, Universidad de Málaga
\end{center}

\bigskip

\begin{center}
	\Large \scshape
	\textbf{\tfgtitlenameENG}
\end{center}

\bigskip \bigskip \bigskip

\begin{minipage}{\textwidth}

\textbf{Author:} \tfgauthorname

\medskip

\textbf{Supervisor:} \tfgtutorname

\medskip

\textbf{Co-supervisor:} José Manuel Cano García

\medskip

\textbf{Department:} Departamento Tecnología Electrónica

\medskip

\textbf{Degree:} Grado en Ingeniería de sistemas electrónicos

\medskip

\textbf{Keywords:} CES, ECG, Real-time QRS detection

\bigskip \bigskip


\end{minipage}

\begin{center}
	\textbf{Abstract}
\end{center}

The project details the process of developing a benchmark environment, to carry out efficiency and reliability measurements, of heartbeat detection and classification algorithms in real time. Finally managing to create a functional prototype, following the guidelines of the UNE-EN 60601-2-47: 2002 standard.
\blankpage
