%%%%%%%%%%%%%%%%%%%%%%%%%%%%%%%%%%%%%%%%%%%%%%%%%%%%%%%%%%%%%%%%%%%
%%% Documento LaTeX 																						%%%
%%%%%%%%%%%%%%%%%%%%%%%%%%%%%%%%%%%%%%%%%%%%%%%%%%%%%%%%%%%%%%%%%%%
% Título:		Capítulo 3
% Autor:  	Ignacio Moreno Doblas
% Fecha:  	2014-02-01, actualizado 2019-11-11
% Versión:	0.5.0
%%%%%%%%%%%%%%%%%%%%%%%%%%%%%%%%%%%%%%%%%%%%%%%%%%%%%%%%%%%%%%%%%%%
% !TEX root = A0.MiTFG.tex

\section{Quinta iteración: Pulido}
    \subsection{Resumen}
    
        Una vez teniendo el proyecto funcionando con todas las funciones básicas implementadas ha llegado el momento de frenar el desarrollo de nuevas funcionalidades y echar la vista atrás con el fin de mejorar el acabado general del proyecto.
        
        En esta iteración se llevarán a cabo tareas relacionadas con facilitar la lectura del código y mejorar el aspecto visual y usabilidad de la aplicación de usuario.
    
    \subsection{Requisitos}
    
        A diferencia de las anteriores, en esta iteración no se satisface ningún requisito en particular, pero igualmente se puede dividir el trabajo en las siguientes tareas.
        
        \begin{enumerate}
            \item Limpieza de código y asegurar coherencia en la nomenclatura.
            \item Añadir comentarios necesarios para facilitar la lectura del código
            \item Establecer coherencia de idioma en el panel de usuario.
            \item Añadir al panel de usuario etiquetas y separadores para organizar y explicar la interfaz.
            \item Mejorar la apariencia visual del panel en la medida de lo posible.
            \item Añadir marcadores al código para tareas restantes y mejoras futuras. 
        \end{enumerate}
    \subsection{Desarrollo}
    
        Para la primera tarea se recorren los diferentes ficheros del código examinando con detalle el estilo de escritura utilizado, para estandarizar todo el código. Aunque no se ha comentado previamente en el proyecto existen convenciones a la hora de escribir código, normalmente estas convenciones se siguen por costumbre sin prestar demasiada atención, pero en esta iteración se ha decidido asegurar la coherencia de la nomenclatura empleando tres convenciones: la primera,  ``lowerCamelCase'' para las variables, ya sean públicas o privadas, y para los tipos, acompañado de ``\char`_t''; en segundo lugar, ``UpperCamelCase'' para las funciones; por último,  ``SCREAMING\char`_SNAKE\char`_CASE'' para la declaración de constantes.
        
        Además debe asegurarse de que todo el código y comentarios mantengan una coherencia en cuanto al lenguaje utilizado, el escogido para este proyecto es el inglés por ser el lenguaje dominante dentro del sector, por ello a la vez que se ha repasado el estilo de escritura si se ha detectado algún comentario o elemento en otro idioma, se ha sustituido.
    \subsection{Pruebas}
    \subsection{Conclusiones}