%%%%%%%%%%%%%%%%%%%%%%%%%%%%%%%%%%%%%%%%%%%%%%%%%%%%%%%%%%%%%%%%%%%
%%% Documento LaTeX 																						%%%
%%%%%%%%%%%%%%%%%%%%%%%%%%%%%%%%%%%%%%%%%%%%%%%%%%%%%%%%%%%%%%%%%%%
% Título:		Capítulo 3
% Autor:  	Ignacio Moreno Doblas
% Fecha:  	2014-02-01, actualizado 2019-11-11
% Versión:	0.5.0
%%%%%%%%%%%%%%%%%%%%%%%%%%%%%%%%%%%%%%%%%%%%%%%%%%%%%%%%%%%%%%%%%%%
% !TEX root = A0.MiTFG.tex

\section{Quinta iteración: Consideraciones de diseño}
    \subsection{Resumen}
    
        Una vez teniendo el proyecto funcionando con todas las funciones básicas implementadas ha llegado el momento de frenar el desarrollo de nuevas funcionalidades y echar la vista atrás con el fin de mejorar el acabado general del proyecto.
        
        En esta iteración se llevarán a cabo tareas relacionadas con facilitar la lectura del código y mejorar el aspecto visual y usabilidad de la aplicación de usuario.
    
    \subsection{Requisitos}
    
        A diferencia de las anteriores, en esta iteración no se satisface ningún requisito en particular, pero igualmente se puede dividir el trabajo en las siguientes tareas.
        
        \begin{enumerate}
            \item Limpieza de código y asegurar coherencia en la nomenclatura.
            \item Añadir comentarios necesarios para facilitar la lectura del código y su mantenimiento.
            \item Mejorar la apariencia visual del panel y mejorar su usabilidad en la medida de lo posible.
        \end{enumerate}
    \subsection{Desarrollo}
    
        Para la primera tarea, \textit{revisar la limpieza del código y su coherencia}, se recorren los diferentes ficheros del código examinando con detalle el estilo de escritura utilizado, para estandarizar todo el código. Aunque no se ha comentado previamente en el proyecto, existen convenciones a la hora de escribir código que se han seguido durante el desarrollo, sin embargo en esta iteración se ha decidido asegurar el cumplimiento de éstas revisando todo lo desarrollado, empezando por la coherencia de la nomenclatura, se han empleado tres convenciones de estilo para diferenciar los diferentes elementos del código: la primera,  ``lowerCamelCase'' para las variables, ya sean públicas o privadas, y para los tipos, acompañado de ``\char`_t''; en segundo lugar, ``UpperCamelCase'' para las funciones; por último,  ``SCREAMING\char`_SNAKE\char`_CASE'' para la declaración de constantes.
        
        Además debe asegurarse de que todo el código y comentarios mantengan una coherencia en cuanto al idioma utilizado. El idioma escogido para este proyecto es el inglés, ya que es el idioma dominante dentro del sector, por ello a la vez que se ha repasado el estilo de escritura, si se ha detectado algún comentario o elemento en otro idioma, se ha sustituido.
        
       \textit{Comentarios explicativos} se han añadido a las declaraciones y demás elementos del código con el objetivo de facilitar la comprensión del mismo, especialmente en las zonas destinadas a ser modificadas por el usuario final. Además se han añadido comentarios referidos a posibles mejoras y correcciones que se podrían implementar en un futuro, por si el sistema sigue desarrollándose una vez concluido este proyecto. Minimizando así el tiempo de estudio previo a comenzar a realizar modificaciones.
       
       Para \textit{mejorar el apartado visual del panel} se han redistribuido los elementos ya expuestos en este proyecto agrupándolos por secciones. Para \textit{mejorar la usabilidad} se han incluido textos descriptivos detallando a que corresponde cada elemento y una nueva funcionalidad que permite detener el envío de una señal para realizar el análisis de otra sin la necesidad de reiniciar el panel. La nueva disposición de los elementos puede observarse en las imagenes (TODO: las dos referencias)
       
       (TODO: Las dos imagenes)
       
    \subsection{Pruebas}
    \subsection{Conclusiones}