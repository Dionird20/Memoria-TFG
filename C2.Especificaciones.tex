%%%%%%%%%%%%%%%%%%%%%%%%%%%%%%%%%%%%%%%%%%%%%%%%%%%%%%%%%%%%%%%%%%%
%%% Documento LaTeX 																						%%%
%%%%%%%%%%%%%%%%%%%%%%%%%%%%%%%%%%%%%%%%%%%%%%%%%%%%%%%%%%%%%%%%%%%
% Título:		Capítulo 2
% Autor:  	Ignacio Moreno Doblas
% Fecha:  	2014-02-01, actualizado 2019-11-11
% Versión:	0.5.0
%%%%%%%%%%%%%%%%%%%%%%%%%%%%%%%%%%%%%%%%%%%%%%%%%%%%%%%%%%%%%%%%%%%
% !TEX root = A0.MiTFG.tex
\chapterbegin{Especificaciones}
%\label{chp:Utiliz}
\minitoc
\clearpage

\section{Diagramas}

En este apartado se engloban todos los diagramas realizados a la hora de definir el proyecto y su alcance.

    \subsection{Contexto}
     
    En la figura \ref{fig:CtxDiagram} se puede observar el entorno circundante al proyecto, representado por un diagrama de contexto. El bloque correspondiente al proyecto en sí es el denominado como ``Entorno pruebas de algoritmos ECG'', los otros bloques representan elementos relacionados con los que se interacciona de una forma u otra. 

    Este diagrama resulta de bastante ayuda a la hora de establecer los casos de uso que serán detallados más adelante, pues al visualizar de un solo vistazo todo el entorno circundante al producto se pueden descubrir nuevas interacciones que habrían sido ignoradas de otra forma.        

    \begin{figure}[H]  
        \centering
            \includegraphics[width =\linewidth]{figuras/ContextDiagram.pdf}
        \caption{Diagrama de contexto}
        \label{fig:CtxDiagram}
    \end{figure}

    \subsection{Casos de uso}

    Una vez establecido el contexto en el que se desenvuelve el proyecto se pueden establecer los casos de uso. De nuevo se realiza un esquema para poder visualizarlos como conjunto. Este esquema puede verse en la figura \ref{fig:UseCasesDiagram}, en él se muestran las acciones que podría llevar a cabo el usuario una vez el proyecto esté finalizado, aunque será tarea futura determinar cuales gozan de mayor prioridad, pues no todas son iguales de fundamentales a la hora de llevar a cabo el objetivo principal del proyecto.

    \begin{figure}[H]  
        \centering
            \includegraphics[width =0.7\linewidth]{figuras/UseCasesDiagram.pdf}
        \caption{Diagrama de casos de uso}
        \label{fig:UseCasesDiagram}
    \end{figure}

    Estando aclaradas estas interacciones ya resulta posible comenzar a establecer los requisitos del proyecto, sin embargo primero se debería establecer conceptualmente el flujo de trabajo que se llevará a cabo para el testeo de los algoritmos.

    \subsection{Flujo}

    De forma muy simplificada se puede observar en la figura \ref{fig:BlocksDiagram} el flujo de trabajo del entorno de pruebas. Este diagrama es útil para tener en mente en todo momento el objetivo del proyecto en su versión más simplificada.

    \begin{figure}[H]  
        \centering
            \includegraphics[width =\linewidth]{figuras/BlocksDiagram.pdf}
        \caption{Diagrama de bloques}
        \label{fig:BlocksDiagram}
    \end{figure}
    \clearpage


\section{Requisitos generales}

    Con la ayuda de los diagramas realizados se establecen los siguientes requisitos para cumplir con el objetivo establecido:
     
    \begin{scriptsize}
    \begin{longtable}{|p{0.05\linewidth}|p{0.21\linewidth}|p{0.3\linewidth}|p{0.3\linewidth}|p{0.03\linewidth}|}
        \caption{Requisitos iniciales. \label{tab:Requisitos}} \\
        \hline
        ID & Requisito & Descripción & Prueba & Prio \\ \hline
        \endfirsthead
        %
        \endhead
        %
        \hline
        \endfoot
        %
        \endlastfoot
        %
        1 & Panel de control & El proyecto debe disponer de un panel de usuario intuitivo para llevar a cabo las pruebas &  & 1 \\ \hline
        1.1 & Introducir señal de prueba & Al panel de control se le debe proveer una señal de ECG formateada específicamente para poder probar el algoritmo deseado & Introducir un ECG extraído de alguna base de datos y comprobar que los valores de las muestras enviados se corresponden & 1 \\ \hline
        1.2     & Observar la señal procesada & Al panel de control se le debe proveer de una gráfica de salida que muestre la señal procesada. & Observar las gráfica de salida y comprobar que los datos dibujados se corresponden con los de la lectura & 2 \\ \hline 
        1.3     & Resultados & El panel de control debe ser capaz de mostrar al usuario los siguientes elementos: &  &  \\ \hline
        1.3.1   & Frecuencia cardiaca instantánea & El panel deberá mostrar la frecuencia cardíaca instantánea medida por el algoritmo a testear. & Comprobar que el dato enviado por la aplicación de pruebas y el recibido por el panel son el mismo. La precisión de este no es un requisito pues dependerá del algoritmo con el que se esté trabajando. & 1 \\ \hline
        1.3.2   & Picos detectados & El panel deberá mostrar información sobre los picos R detectados y sus posiciones, para facilitar la evaluación del algoritmo & Al igual que la prueba anterior, se debe garantizar la coherencia de los datos entre la aplicación y el panel de usuario. La validez de estos dependerá del algoritmo. & 1 \\ \hline
        1.3.3   & Utilización del procesador & El panel deberá mostrar el uso de CPU que corresponde al proceso del algoritmo de la forma más aproximada posible & Introducir una función con una duración conocida y realizar la estimación. & 1 \\ \hline
        1.3.4   & Utilización de memoria & El panel deberá mostrar el uso de memoria en cada momento de la forma más aproximada posible & Introducir una función que consuma una memoria conocida y realizar la medición & 1 \\ \hline
        1.3.5   & Fiabilidad del algoritmo visual & El panel deberá mostrar la fiabilidad del algoritmo de forma visual superponiendo los picos detectados con la señal en la gráfica. & Asegurar que los datos recibidos y dibujados son los mismos. La validez de estos dependen del algoritmo. & 1 \\ \hline
        1.3.6   & Fiabilidad del algoritmo contrastada & El panel deberá mostrar la fiabilidad del algoritmo en forma de tasa de falsos negativos y positivos. & Contrastar los resultados con las anotaciones de la base de datos. & 2 \\ \hline
        1.4     & Resolución muestras & El panel deberá poder configurar la resolución empleada para las muestras en la aplicación, ofreciendo al usuario cierto control sobre la cantidad de muestras simultaneas con las que su algoritmo podrá trabajar. & Comprobar la resolución de los valores con un inspector en modo \textit{debug}.  & 2 \\ \hline
        2       & Dispositivo de pruebas & El sistema debe disponer de un dispositivo encargado de simular el comportamiento de un supervisor de eventos cardíacos que emplee el algoritmo de detección de tasa cardíaca deseado. &  & 1 \\ \hline
        2.1     & Protocolo de comunicación & La aplicación debe ser capaz de recibir los datos desde el panel de usuarios y devolver los resultados procesados. & El correcto funcionamiento del resto de requisitos es prueba suficiente para este. & 1 \\ \hline
        2.2     & Entrada de datos en tiempo real & La aplicación debe enviar y recibir los datos en tiempo real para emular las condiciones de funcionamiento de un SEC. & Comprobar que la aplicación es capaz de recibir datos paulatinamente y devolverlos. & 1 \\ \hline
        2.2.1   & Simular la entrada de datos por el ADC & La aplicación debe simular la adquisición de datos mediante un conversor analógico digital. Emulando el funcionamiento de un supervisor real. & Comprobar que el acceso a esos datos está controlado con un \textit{timer}, como si del ADC se tratase. & 2 \\ \hline
        2.3     & Interfaz para el algoritmo & La aplicación debe disponer de una interfaz, dentro de la cual se pueda encapsular el algoritmo de detección de tasa cardíaca que el usuario desee probar. & Implementar dos algoritmos diferentes sin necesidad de modificar nada fuera de la interfaz. & 1 \\ \hline
        2.4     & Resolución muestras & La aplicación debe ser capaz de modificar la resolución empleada para las muestras en la aplicación, ofreciendo al usuario cierto control sobre la cantidad de muestras simultaneas con las que su algoritmo podrá trabajar. & Comprobar que el máximo de muestras almacenadas en el dispositivo varían en función de la calidad seleccionada en el panel de control. & 2 \\ \hline
    \end{longtable}        
    \end{scriptsize}
    
    Adicionalmente y para validar el sistema, se debe diseñar e implementar un algoritmo sencillo de detección de QRS. 
    \clearpage

\section{Consideraciones de diseño}

    Durante el desarrollo se van a seguir una serie de convenciones a la hora de escribir código. Empezando por la coherencia de la nomenclatura, se han empleado tres convenciones de estilo para diferenciar los diferentes elementos del código: la primera,  \textit{lowerCamelCase} para las variables, ya sean públicas o privadas, y para los tipos, acompañado de \textit{\char`_t}; en segundo lugar, \textit{UpperCamelCase} para las funciones; por último,  \textit{SCREAMING\char`_SNAKE\char`_CASE} para la declaración de constantes.
        
    Además debe asegurarse de que todo el código y comentarios mantengan una coherencia en cuanto al idioma utilizado. El idioma escogido para este proyecto es el inglés, ya que es el idioma dominante dentro del sector y resulta en un producto más internacional. También se va a emplear el inglés como idioma para todo el texto que se presente en el panel de usuario.

\section{Metodología}

    Para la realización de este proyecto se ha decidido emplear una metodología de desarrollo ágil, en lugar de una metodología más tradicional como podría ser un desarrollo en cascada. El carácter iterativo e incremental de este tipo de diseño permite reaccionar más rápidamente a los cambios de diseño o nuevas necesidades que se encuentren durante el desarrollo.

    Uno de los marcos de trabajo ágiles más empleados en el desarrollo de software es el \textit{Scrum}, que será empleado de una forma simplificada para el desarrollo de este proyecto. Dentro del \textit{Scrum} nos encontramos tres figuras (\textit{Product owner}, \textit{Scrum master} y Equipo) sin embargo dado el carácter individual de este proyecto solo se emplearán dos. El \textit{product owner}, representado por los tutores, que revisará el proyecto tras cada iteración y el desarrollador, encargado de organizar y llevar a cabo las iteraciones.

    \begin{figure}[H]  
        \centering
            \includegraphics[width =0.9\linewidth]{figuras/Agile.png}
        \caption{Diagrama conceptual de la metodología ágil.}
        \label{fig:agile}
    \end{figure}

    Una iteración es la unidad fundamental de trabajo en una metodología ágil, y consiste en la adición de un nuevo fragmento de código que añade o mejora alguna función del proyecto, tratando de generar valor en cada iteración. Cada una de estas unidades de trabajo consta de su propia planificación, análisis de requisitos, diseño, codificación y su propia documentación.

    Dada la necesidad de agrupar toda la documentación generada del proyecto en un solo documento, en lugar de generar un fichero de documentación específico para cada iteración se clasificará cada iteración como un apartado del documento. El lector deberá tener en cuenta pues que los apartados referentes a las iteraciones del proyecto gozarán de una temporalidad y orden explícitos.
    
    Respecto a la metodología ágil, otro apartado imprescindible consiste en la realización de pruebas antes de finalizar cada iteración, asegurando el funcionamiento de cada pedazo de código añadido evitando la propagación de errores a lo largo del proyecto.
    
    Dada la naturaleza dual del proyecto se empleará una metodología de pruebas diferente para cada parte del proyecto. El panel de usuario será comprobado mediante \textit{Unit Testing} o pruebas unitarias. Para el dispositivo de pruebas se realizaran los tests con la ayuda de un \textit{debugger} y puntos de ruptura, además, mediante la coherencia de los datos retornados al panel se puede comprobar el correcto funcionamiento del sistema.

\clearpage

\chapterend{}
