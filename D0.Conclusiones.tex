% !TEX root = A0.MiTFG.tex

\chapterbegin{Conclusiones y líneas futuras}

\section{Requisitos finales}

A continuación se adjunta una tabla con los requisitos completos del proyecto, incluidos los que han surgido durante el desarrollo, y en ella se indica cuales se han satisfecho y cuales no.

 \begin{scriptsize}
    \begin{longtable}{|p{0.05\linewidth}|p{0.20\linewidth}|p{0.25\linewidth}|p{0.25\linewidth}|p{0.03\linewidth}|p{0.02\linewidth}|}
        \hline
        ID & Requisito & Descripción & Prueba & Prio & Fin\\ \hline
        \endfirsthead
        %
        \endhead
        %
        \hline
        \endfoot
        %
        \endlastfoot
        %
        1 & Panel de control & El proyecto debe disponer de un panel de usuario intuitivo para llevar a cabo las pruebas &  & 1 & \\ \hline
        1.1 & Introducir señal de prueba & Al panel de control se le debe proveer una señal de ECG formateada específicamente para poder probar el algoritmo deseado & Introducir un ECG extraído de alguna base de datos y comprobar que los valores de las muestras enviados se corresponden & 1 & Si\\ \hline
        1.2     & Observar la señal procesada & Al panel de control se le debe proveer de una gráfica de salida que muestre la señal procesada. & Observar las gráfica de salida y comprobar que los datos dibujados se corresponden con los de la lectura & 2 & Si\\ \hline 
        1.3     & Resultados & El panel de control debe ser capaz de mostrar al usuario los siguientes elementos: &  &  & \\ \hline
        1.3.1   & Frecuencia cardiaca instantánea & El panel deberá mostrar la frecuencia cardíaca instantánea medida por el algoritmo a testear. & Comprobar que el dato enviado por la aplicación de pruebas y el recibido por el panel son el mismo. La precisión de este no es un requisito pues dependerá del algoritmo con el que se esté trabajando. & 1 & Si\\ \hline
        1.3.2   & Picos detectados & El panel deberá mostrar información sobre los picos R detectados y sus posiciones, para facilitar la evaluación del algoritmo & Al igual que la prueba anterior, se debe garantizar la coherencia de los datos entre la aplicación y el panel de usuario. La validez de estos dependerá del algoritmo. & 1 & Si\\ \hline
        1.3.3   & Utilización del procesador & El panel deberá mostrar el uso de CPU que corresponde al proceso del algoritmo de la forma más aproximada posible & Introducir una función con una duración conocida y realizar la estimación. & 1 & Si\\ \hline
        1.3.4   & Utilización de memoria & El panel deberá mostrar el uso de memoria en cada momento de la forma más aproximada posible & Introducir una función que consuma una memoria conocida y realizar la medición & 1 & Si\\ \hline
        1.3.5   & Fiabilidad del algoritmo visual & El panel deberá mostrar la fiabilidad del algoritmo de forma visual superponiendo los picos detectados con la señal en la gráfica. & Asegurar que los datos recibidos y dibujados son los mismos. La validez de estos dependen del algoritmo. & 1 & Si\\ \hline
        1.3.6   & Fiabilidad del algoritmo contrastada & El panel deberá mostrar la fiabilidad del algoritmo en forma de tasa de falsos negativos y positivos. & Contrastar los resultados con las anotaciones de la base de datos. & 2 & Si\\ \hline
        1.4     & Resolución muestras & El panel deberá poder configurar la resolución empleada para las muestras en la aplicación, ofreciendo al usuario cierto control sobre la cantidad de muestras simultaneas con las que su algoritmo podrá trabajar. & Comprobar la resolución de los valores y la cantidad de muestras totales a procesar con un inspector en modo debug.  & 2 & No\\ \hline
        1.5 & Selección de datos & El panel deberá proporcionar al usuario un mecanismo por el cual pueda seleccionar de forma facil e intuitiva el conjunto de muestras a emplear sobre sus pruebas & Comprobar que las muestras son tomadas y que se puede modificar la entrada de forma intuitiva. & 1 & Si \\ \hline
        1.6 & Mecanismo de mensajes & El panel deberá proporcionar al usuario información de los errores que se produzcan, además de manejarlos adecuadamente. & Forzar todos los errores conocidos y observar que la información mostrada al usuario es coherente & 2 & No \\ \hline
        2       & Dispositivo de pruebas & El sistema debe disponer de un dispositivo encargado de simular el comportamiento de un supervisor de eventos cardíacos que emplee el algoritmo de detección de tasa cardíaca deseado. &  & 1 & \\ \hline
        2.1     & Protocolo de comunicación & La aplicación debe ser capaz de recibir los datos desde el panel de usuarios y devolver los resultados procesados. & El correcto funcionamiento del resto de requisitos es prueba suficiente para este. & 1 & Si\\ \hline
        2.2     & Entrada de datos en tiempo real & La aplicación debe enviar y recibir los datos en tiempo real para emular las condiciones de funcionamiento de un SEC. & Comprobar que la aplicación es capaz de recibir datos paulatinamente y devolverlos. & 1 & Si\\ \hline
        2.2.1   & Simular la entrada de datos por el ADC & La aplicación debe simular la adquisición de datos mediante un conversor analógico digital. Emulando el funcionamiento de un supervisor real. & Comprobar que el acceso a esos datos está controlado con un Timer, como si del ADC se tratase. & 2 & No\\ \hline
        2.3     & Interfaz para el algoritmo & La aplicación debe disponer de una interfaz, dentro de la cual se pueda encapsular el algoritmo de detección de tasa cardíaca que el usuario desee probar. & Implementar dos algoritmos diferentes sin necesidad de modificar nada fuera de la interfaz. & 1 & Si\\ \hline
        2.4     & Resolución muestras & La aplicación debe ser capaz de modificar la resolución empleada para las muestras en la aplicación, ofreciendo al usuario cierto control sobre la cantidad de muestras simultaneas con las que su algoritmo podrá trabajar. & Comprobar que el máximo de muestras almacenadas en el dispositivo varían en función de la calidad seleccionada en el panel de control. & 2 & No\\ \hline
        2.5 & Aislamiento del procesado de señal & La implementación del algoritmo del usuario debe quedar encapsulada para facilitar las mediciones.  & Comprobar que el mecanismo de encapsulado se está empleando correctamente.  & 2 & Si \\ \hline
        \caption{Requisitos Finales}
        \label{tab:RequisitosFinales}
    \end{longtable}        
    \end{scriptsize}

\section{Conclusiones}

Tras la finalización del proyecto, podemos decir con satisfacción que se han cumplido todos los objetivos que se plantearon al comienzo del desarrollo, tanto los referentes al sistema a desarrollar como los relacionados con el aprendizaje del autor de este TFG.

\textit{Respecto al  sistema}, se ha obtenido un entorno de pruebas para testear algoritmos de medida de tasa cardiaca instantánea en un entrono portátil de tiempo real. El sistema resultante ha sido verificado en cada iteración de su desarrollo, y, tal y como se planeó, ha sido validado con un algoritmo sencillo de detección de tasa cardiaca usando algunos registros de la MIT-BIH.

En primer lugar, se ha conseguido dotar a un sistema empotrado de las funcionalidades necesarias para que el usuario pueda cargar su propio algoritmo de detección de tasa cardíaca y se ha desarrollado una aplicación funcional, sobre el ordenador, capaz de enviar señales de ECG, extraídas de una base de datos, al algoritmo contenido en el sistema empotrado. En segundo lugar, se han desarrollado  mecanismos para testear el algoritmo, tanto desde el punto de vista de su fiabilidad como de su rendimiento. En el primer caso, contrastando los resultados obtenidos por dicho algoritmo con las anotaciones realizadas por un profesional especializado, y en el segundo, midiendo la duración de su ejecución y la memoria libre de la tarea que lo encapsula. 

Por tanto, se han cumplido todos los requisitos prioritarios que se plantearon para el entorno de pruebas al comienzo del desarrollo, así como algunos de menor prioridad pero de gran relevancia. 

Sin embargo, dos de los requisitos no prioritarios no han llegado a realizarse, principalmente por motivos temporales, ya que se ha entendido que carecían de la suficiente relevancia como para dilatar la realización del proyecto. Ambos requisitos se mencionarán más adelante como posibles líneas futuras de trabajo a seguir. A cambio, en el transcurso del desarrollo se ha visto la necesidad de implementar nuevos requisitos respecto a los planeados relacionados con la mejora de la usabilidad (iteración 1) y de la medida del rendimiento (iteración 2). 

Se debe mencionar también un requisito que no se contempló inicialmente y que se ha mencionado poco durante el desarrollo: la implementación de un mecanismo de control de errores. Si bien el sistema, al ser bastante específico en sus funciones, no es muy dado a tener fallos críticos, pero si sucediera, el usuario tendría poca información para averiguar cual ha sido el problema que lo originó.
 
El proyecto ha seguido tanto la metodología de trabajo como las consideraciones de diseño que se plantearon en su inicio. Respecto a la metodología ágil, el carácter iterativo de esta ha evitado bloqueos en el proyecto al poder revisitar siempre tareas previas de forma ordenada, además se han podido implementar funciones secundarias que no estaban previamente planeadas sin alterar demasiado la planificación inicial.  Respecto a las consideraciones de diseño, se ha intentado hacer un diseño usable del interfaz y diseño mantenible del código, dedicándose la última iteración a mejorar algunos aspectos de ambas, aunque, de nuevo con la limitaciones de tiempo propias de un TFG. 

\textit{Respecto a los objetivos de aprendizaje del autor de este TFG}, se han conseguido alcanzar en gran medida los que se plantearon al comienzo del trabajo. Se ha empleado un sistema operativo de tiempo real en el microcontrolador y se ha usado QT para realizar un diseño de interfaz para el panel de usuario, profundizando así en el uso de algunos conceptos aprendidos en el grado. También, se ha elaborado este documento empleando \LaTeX, cumpliendo así el objetivo de tomar contacto con dicha herramienta y aprender a manejarla.


\section{Lineas futuras}

Como se comentaba en el apartado de conclusiones algunos requisitos no se han podido llevar a cabo. 

De entre todos ellos, el control de errores es el más relevante y el que más se echa en falta. Sin duda con más tiempo, esta sería la funcionalidad más importante a añadir a continuación, pues mejora la robustez del sistema y la experiencia de usuario a partes iguales.

Por otro lado, modificar la resolución de las muestras con las que trabaja el microcontrolador, es uno de los requisito que no ha sido llevado a cabo, que hubiera aportado la posibilidad de cubrir un rango mayor de la señal a procesar simultáneamente.

Simular la entrada de datos por el ADC del microcontroladores otro de los requisitos no implementado,  ya que no incluirlo no varía los resultados ni le resta validez al método implementado, pero que sí podía aportar más realismo al sistema, a la vez que sería el primer paso para convertir el sistema desarrollado en un simulador de eventos cardiacos.

Otra cuestión importante en la que se debería invertir más tiempo de desarrollo es en la interfaz de usuario del panel. Tratando de mejorar su diseño en términos de usabilidad, ordenando y bloqueando los elementos para mejorar la navegación. Ayudando así a que la aplicación dependiese menos de su manual de usuario y haciendo más satisfactorio su uso. Por supuesto, habría que considerar el diseño e implementación de pruebas de usuario que validaran este futuro diseño del interfaz.

\chapterend
