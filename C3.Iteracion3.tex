%%%%%%%%%%%%%%%%%%%%%%%%%%%%%%%%%%%%%%%%%%%%%%%%%%%%%%%%%%%%%%%%%%%
%%% Documento LaTeX 																						%%%
%%%%%%%%%%%%%%%%%%%%%%%%%%%%%%%%%%%%%%%%%%%%%%%%%%%%%%%%%%%%%%%%%%%
% Título:		Capítulo 3
% Autor:  	Ignacio Moreno Doblas
% Fecha:  	2014-02-01, actualizado 2019-11-11
% Versión:	0.5.0
%%%%%%%%%%%%%%%%%%%%%%%%%%%%%%%%%%%%%%%%%%%%%%%%%%%%%%%%%%%%%%%%%%%
% !TEX root = A0.MiTFG.tex

\section{Tercera iteración: Prototipo}
    \subsection{Resumen}
        
        Con esta iteración se pretende añadir todo lo referente a las mediciones de rendimiento del algoritmo. Tratando de concluir el prototipo inicial. Esta iteración aún se centrará exclusivamente en la funcionalidad del proyecto dejando de lado todas las cuestiones referentes a diseño de interfaces y experiencia de usuario para iteraciones posteriores.
        
    \subsection{Requisitos}

    Los requisitos a realizar en esta iteración son el 1.3.3 y el 1.3.4 de la tabla inicial (Tabla \ref{tab:Requisitos}) junto al requisito 2.5 de la tabla completa (TODO: Referencia tabla final) añadido durante la iteración anterior. Para ello se divide el trabajo de la iteración en las siguientes tareas:

        \begin{enumerate}
            \item Aislamiento del procesado de señales en una tarea exclusiva.
            \item Extraer las medidas de rendimiento del sistema operativo.
            \item Mostrar dichas medidas en el panel de usuario de QT.
            \item Comparar con las anotaciones de la señal.
        \end{enumerate}
        
    \subsection{Desarrollo}
        
    \subsection{Pruebas}
        
    \subsection{Conclusiones}