%%%%%%%%%%%%%%%%%%%%%%%%%%%%%%%%%%%%%%%%%%%%%%%%%%%%%%%%%%%%%%%%%%%
%%% Documento LaTeX 																						%%%
%%%%%%%%%%%%%%%%%%%%%%%%%%%%%%%%%%%%%%%%%%%%%%%%%%%%%%%%%%%%%%%%%%%
% Título:		Capítulo 3
% Autor:  	Ignacio Moreno Doblas
% Fecha:  	2014-02-01, actualizado 2019-11-11
% Versión:	0.5.0
%%%%%%%%%%%%%%%%%%%%%%%%%%%%%%%%%%%%%%%%%%%%%%%%%%%%%%%%%%%%%%%%%%%
% !TEX root = A0.MiTFG.tex

\section{Tercera iteración: Prototipo}
    \subsection{Resumen}
        
        Con esta iteración se pretende añadir todo lo referente a las mediciones de rendimiento del algoritmo. Tratando de concluir el prototipo inicial. Esta iteración aún se centrará exclusivamente en la funcionalidad del proyecto dejando de lado todas las cuestiones referentes a diseño de interfaces y experiencia de usuario para iteraciones posteriores.
        
    \subsection{Requisitos}

        Los requisitos a realizar en esta iteración son el 1.3.3 y el 1.3.4 de la tabla inicial (Tabla \ref{tab:Requisitos}) junto al requisito 2.5 de la tabla completa (TODO: Referencia tabla final) añadido durante la iteración anterior. Para ello se divide el trabajo de la iteración en las siguientes tareas:

        \begin{enumerate}
            \item Aislamiento del procesado de señales en una tarea exclusiva.
            \item Extraer las medidas de rendimiento del sistema operativo.
            \item Mostrar dichas medidas en el panel de usuario de QT.
            \item Comparar los resultados del algoritmo con las anotaciones de la señal.
            \begin{enumerate}
                \item Extraer las anotaciones de la base de datos y añadirlas al fichero de entrada.
                \item Implementar la lectura de las anotaciones en el panel de usuario.
                \item Crear un modelo de almacenamiento que facilite la implementación de la norma (TODO: Norma de las evaluaciones)
                \item Implementar las ecuaciones descritas en la norma (TODO: Norma de las evaluaciones)
                \item Mostrar al usuario los resultados de la validación del algoritmo.
            \end{enumerate}
        \end{enumerate}
        
    \subsection{Desarrollo}
        
        El procesado de la señal se ha movido de la rutina encargada de manejar las comunicaciones, donde se alojaba inicialmente de forma provisional, a una tarea propia, desde la que resulta más sencillo calcular el tiempo de procesado y determinar los recursos que están siendo consumidos por la misma. Además independizar estas dos tareas también ayuda a descongestionar las comunicaciones entre el dispositivo y el panel de control. Sin embargo con la adición de las funciones necesarias para el funcionamiento de la tarea se carga de contenido el fichero que finalmente será editable por el usuario, por este motivo, se ha decidido crear un fichero dedicado exclusivamente a las funciones del usuario llamado UserEntry.c y su correspondiente fichero de encabezado UserEntry.h, facilitando así la implementación de los algoritmos a evaluar. 
        
        Con los cambios incurridos en esta tarea, el diagrama de flujo del ciclo de trabajo del dispositivo de pruebas queda conforme a la figura (TODO: Referencia diagrama de flujo TIVA).
        
        (TODO: Añadir un diagrama de flujo de los procesos de la TIVA)
        
        Para la realización de la segunda tarea se ha empleado un Timer de alta frecuencia del dispositivo para contar el número de ciclos de reloj empleados en llevar a cabo cada ciclo de ejecución del algoritmo. El Timer empleado es el Timer1 configurado en modo cuenta periódica de 32 bits ascendente. Para realizar la medida al comienzo de cada ciclo de procesado se le asigna el valor 0 y cuando el procesado concluye se consulta la cuenta del temporizador, siendo esta equivalente al numero de ciclos de reloj transcurridos en el proceso. Si este valor se divide entre la frecuencia del mismo se obtiene una medida del tiempo empleado. La implementación puede observarse en la figura (TODO: Referencia a la figura)
        
        (TODO: Adjuntar figura implementación Timer)
        
        Para mostrar las medidas de rendimiento al usuario en el panel primero era necesario enviarlas, y para ello se ha decidido emplear otro paquete diferente al de los resultados del algoritmo. De esta forma se evita sobrecargar un paquete que ya es de por sí bastante grande y además se consigue independencia previendo que en alguna iteración posterior se puedan añadir nuevos indicadores de rendimiento, que pueden incluso no depender de la ejecución del algoritmo. Respecto a la visualización de los datos, de momento solo se han añadido las salidas de texto correspondiente con la mínima explicación necesaria, posponiendo los detalles de interfaz de usuario para más adelante.
        
    \subsection{Pruebas}
        
    \subsection{Conclusiones}
    
    (TODO: La tarea 4 ha resultado ser mucho más compleja de lo esperado inicialmente añadirla con detalle a los requisitos)