%%%%%%%%%%%%%%%%%%%%%%%%%%%%%%%%%%
% Página de resumen del proyecto %
%%%%%%%%%%%%%%%%%%%%%%%%%%%%%%%%%%
% !TEX root = A0.MiTFG.tex

\pagestyle{fancy}
\renewcommand{\headrulewidth}{0pt}
\addstarredchapter{Resumen}

\begin{center}
	\scshape
	E.T.S. de Ingeniería de Telecomunicación, Universidad de Málaga 
\end{center}

\bigskip

\begin{center}
	\Large \scshape
	\textbf{\tfgtitlename}
\end{center}

\bigskip \bigskip \bigskip

\begin{minipage}{\textwidth}

\textbf{Autor:} \tfgauthorname

\medskip

\textbf{Tutor:} \tfgtutorname

\medskip

\textbf{Cotutor:} José Manuel Cano García

\medskip

\textbf{Departamento:} Departamento Tecnología Electrónica

\medskip

\textbf{Titulación:} Grado en Ingeniería de sistemas electrónicos

\medskip

\textbf{Palabras clave:} SEC, ECG, Detección QRS en tiempo real

\bigskip \bigskip


\end{minipage}

\begin{center}
	\textbf{Resumen}
\end{center}

El proyecto detalla el proceso de desarrollo de un entorno de pruebas, para realizar mediciones de eficacia y fiabilidad, de algoritmos de detección y clasificación de latidos en tiempo real. Logrando finalmente crear un prototipo funcional siguiendo las recomendaciones de la norma UNE-EN 60601-2-47:2002. 

\blankpage
