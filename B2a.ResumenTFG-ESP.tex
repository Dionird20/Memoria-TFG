%%%%%%%%%%%%%%%%%%%%%%%%%%%%%%%%%%
% Página de resumen del proyecto %
%%%%%%%%%%%%%%%%%%%%%%%%%%%%%%%%%%
% !TEX root = A0.MiTFG.tex

\pagestyle{fancy}
\renewcommand{\headrulewidth}{0pt}
\addstarredchapter{Resumen}

\begin{center}
	\scshape
	E.T.S. de Ingeniería de Telecomunicación, Universidad de Málaga 
\end{center}

\bigskip

\begin{center}
	\Large \scshape
	\textbf{\tfgtitlename}
\end{center}

\bigskip \bigskip \bigskip

\begin{minipage}{\textwidth}

\textbf{Autor:} \tfgauthorname

\medskip

\textbf{Tutor:} \tfgtutorname

\medskip

\textbf{Cotutor:} José Manuel Cano García

\medskip

\textbf{Departamento:} Departamento Tecnología Electrónica

\medskip

\textbf{Titulación:} Grado en Ingeniería de sistemas electrónicos

\medskip

\textbf{Palabras clave:} Supervisor de Eventos Cardiacos, Electrocardiografía, ECG, Detección QRS en tiempo real, Qt, Tiva C Series (TM4C123GXL).

\bigskip \bigskip


\end{minipage}

\begin{center}
	\textbf{Resumen}
\end{center}

Los supervisores de eventos cardiacos monitorizan, en tiempo real, las anomalías en el registro de Electrocardiografía (ECG) de los pacientes que los portan en su vida diaria, permitiendo almacenar, e incluso trasmitir, los segmentos ECG anormales. Así, deben ser diseñados tanto para minimizar su consumo y coste, como para presentar una alta fiabilidad en la detección en tiempo real de las anomalías cardiacas. La detección de los latidos cardiacos contenidos en un segmento de ECG es, normalmente, el paso previo para clasificarlo como anormal. En el presente Trabajo Fin de Grado se ha desarrollado un entorno de pruebas para evaluar de una forma realista el desempeño de los algoritmos de detección de latidos en un sistema empotrado con capacidades de cómputo limitadas.  El sistema está compuesto por una TIVA serie C, sobre la que se ejecuta el algoritmo a probar, conectada con un ordenador que permite la interacción con el usuario mediante una interfaz gráfica basada en Qt. Para su validación se ha usado un algoritmo básico de detección de latidos, implementado también en este TFG, la base de datos de ECG MIT-BIH y las recomendaciones de la norma UNE-EN 60601-2-47 para ensayo de monitores cardiacos portátiles.


\blankpage
