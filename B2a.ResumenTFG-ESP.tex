%%%%%%%%%%%%%%%%%%%%%%%%%%%%%%%%%%
% Página de resumen del proyecto %
%%%%%%%%%%%%%%%%%%%%%%%%%%%%%%%%%%
% !TEX root = A0.MiTFG.tex

\pagestyle{fancy}
\renewcommand{\headrulewidth}{0pt}
\addstarredchapter{Resumen}

\begin{center}
	\scshape
	E.T.S. de Ingeniería de Telecomunicación, Universidad de Málaga 
\end{center}

\bigskip

\begin{center}
	\Large \scshape
	\textbf{\tfgtitlename}
\end{center}

\bigskip \bigskip \bigskip

\begin{minipage}{\textwidth}

\textbf{Autor:} \tfgauthorname

\medskip

\textbf{Tutor:} \tfgtutorname

\medskip

\textbf{Cotutor:} José Manuel Cano García

\medskip

\textbf{Departamento:} Departamento Tecnología Electrónica

\medskip

\textbf{Titulación:} Grado en Ingeniería de sistemas electrónicos

\medskip

\textbf{Palabras clave:} Supervisor de Eventos Cardiacos, Electrocardiografía, ECG, Detección QRS en tiempo real, Qt, Tiva C Series (TM4C123GXL).

\bigskip \bigskip


\end{minipage}

\begin{center}
	\textbf{Resumen}
\end{center}

La monitorización, en tiempo real, de las anomalías en el registro de electrocardiografía (ECG) de pacientes con patologías cardíacas es la base del funcionamiento de los supervisores de eventos cardíacos. Por ello, deben ser diseñados tanto para minimizar su consumo y coste, como para presentar una alta fiabilidad en la detección en tiempo real de las anomalías cardíacas. 

Habida cuenta que, normalmente, el paso previo para la clasificación de un segmento ECG como anormal es la detección de los latidos cardíacos contenidos en dicho segmento, en el presente Trabajo Fin de Grado se ha desarrollado un entorno de pruebas para realizar mediciones de eficacia y fiabilidad de algoritmos de detección de latidos cardíacos en tiempo real, logrando crear un prototipo funcional siguiendo las recomendaciones de la norma UNE-EN 60601-2-47 para ensayo de monitores cardíacos portátiles.

La validación del sistema desarrollado se ha hecho usando un algoritmo básico de detección de latidos implementado también en este TFG y la base de datos de ECG MIT-BIH.

\blankpage
